%
% Device Driver Development Demystified
% Copyright (C) 2016 Thomas Cort
%
\documentclass{beamer}
\usepackage{listings}
\usetheme{Madrid}

\begin{document}

\title{Device Driver Development Demystified}
%%\subtitle{MINIXCon 2016}
%\author{Thomas Cort \texttt{<tcort@minix3.org>}}
\author{Thomas Cort}
\date{February 1, 2016}
\institute{MINIXCon 2016}

\begin{frame}
\titlepage
\end{frame}

\begin{frame}{whoami(1)}
  Open Source Contributor and Former Google Summer of Code Student
  \begin{itemize}
  \item GSoC 2010 - UNIX domain sockets
  \item GSoC 2011 - software porting and pkgsrc improvements
  \item GSoC 2013 - i2c drivers for the BeagleBone Black
  \item Tested and Imported 90+ programs from NetBSD
  \item Other minor patches
  \end{itemize}
\end{frame}

\begin{frame}{Qualifications}
  I've developed several drivers:
  \begin{itemize}
  \item i2c bus driver for BeagleBone (Black and White) \& BeagleBoard-xM
  \item Weather Cape: SHT21, BMP085, TSL2550
  \item Power Management: TPS65217, TPS65950
  \item RTC, CAT24C256, TDA19988 (basic), GPIO (basic)
  \item i2cscan(8), rebooting, poweroff (BeagleBone)
  \end{itemize}
\end{frame}

\begin{frame}{Goal}
  \centering
  Inspire you to develop more device drivers for Minix
\end{frame}

\begin{frame}{How we're going to get there?}
  \centering
  Show how device drivers are built from concept to code
\end{frame}

\begin{frame}{Developing device drivers on Minix rocks}
  \begin{itemize}
  \item Infinite loops don't make the whole system freeze
  \item Bad pointers don't make the whole system crash
  \item No rebooting needed. Just start and stop your driver
  \item Most drivers are single threaded
  \item Most drivers only handle one request at a time
  \item Messages between drivers are well defined and documented
  \end{itemize}
\end{frame}

\begin{frame}{Questions to ask yourself when choosing a device}
  \begin{itemize}
  \item is it already supported (partially or fully)?
  \item is a similar device already supported?
  \item is anyone else already working on it?
  \item is there documentation available?
  \item is there hardware available?
  \item is the device supported by other FOSS operating systems?
  \item is it possible to use the device on a system supported by Minix?
  \end{itemize}
\end{frame}

\begin{frame}{Where to begin}
  Gather documentation:
  \begin{itemize}
  \item Existing Minix drivers and the Minix wiki
  \item System Reference Manuals (SRM)
  \item Technical Reference Manuals (TRM)
  \item Data Sheets
  \item Example code from the device manufacturer (sdk)
  \item Code from other operating systems (Linux, BSD, Haiku, etc)
  \end{itemize}
\end{frame}

\begin{frame}{Eliminate hardware issues before you write a line of code}
  \begin{itemize}
  \item Start with new or gently used hardware
  \item Test the device with another OS
  \item Test the computer with Minix
  \end{itemize}
\end{frame}

\begin{frame}{External Interface Design}
  Stand on the shoulders of giants!
  \begin{itemize}
  \item If it's a new class of device, follow the lead of NetBSD
  \item If it's an existing class of device, implement the established interface
  \end{itemize}
\end{frame}

\begin{frame}{Case Study}
  \centering
  bmp085 temperature sensor
\end{frame}

\defverbatim[colored]\lstI{
\begin{lstlisting}[language=sh,basicstyle=\ttfamily,keywordstyle=\color{red}]
/usr/src/minix/drivers/sensors/bmp085
├── bmp085.c
├── Makefile
└── README.txt
\end{lstlisting}
}

\begin{frame}{File Layout}
  \lstI
\end{frame}

\defverbatim[colored]\lstI{
\begin{lstlisting}[language=sh,basicstyle=\ttfamily]
.include <bsd.own.mk>

.if ${MACHINE_ARCH} == "earm"
SUBDIR+=        bmp085
SUBDIR+=        sht21
SUBDIR+=        tsl2550
.endif # ${MACHINE_ARCH} == "earm"

.include <bsd.subdir.mk>
\end{lstlisting}
}

\begin{frame}{../Makefile}
  \lstI
\end{frame}


\defverbatim[colored]\lstI{
\begin{lstlisting}[language=sh,basicstyle=\ttfamily]
# Makefile for the bmp085 pressure and temp sensor found on the Weather Cape.
PROG=   bmp085
SRCS=   bmp085.c

DPADD+= ${LIBI2CDRIVER} ${LIBCHARDRIVER} ${LIBSYS}\
        ${LIBTIMERS}
LDADD+= -li2cdriver -lchardriver -lsys -ltimers

CPPFLAGS+=      -I${NETBSDSRCDIR}

.include <minix.service.mk>
\end{lstlisting}
}

\begin{frame}{./Makefile}
  \lstI
\end{frame}

\begin{frame}{./README.txt}
  Provides answers to the following questions:
  \begin{itemize}
  \item what does the driver do?
  \item what is in each source file?
  \item how do I start/stop/test the driver?
  \item are there any limitations?
  \item where can I find more information about this device?
  \end{itemize}
\end{frame}

\defverbatim[colored]\lstI{
\begin{lstlisting}[language=sh,basicstyle=\ttfamily]
service bmp085
{
        ipc SYSTEM RS DS i2c;
};
\end{lstlisting}
}

\begin{frame}{etc/system.conf}
  \lstI
\end{frame}


\defverbatim[colored]\lstI{
\begin{lstlisting}[language=sh,basicstyle=\ttfamily]
./service/bmp085    minix-base
\end{lstlisting}
}

\begin{frame}{distrib/sets/lists/minix-base/md.evbarm}
  \lstI
\end{frame}


\defverbatim[colored]\lstI{
\begin{lstlisting}[language=C,basicstyle=\ttfamily]
int main(int argc, char *argv[]) {
        int r;

        env_setargs(argc, argv);

        r = i2cdriver_env_parse(&bus, &address,
                                      valid_addrs);
        if (r < 0) /* ... */
        else if (r > 0) /* ... */

        sef_local_startup();
        chardriver_task(&bmp085_tab);
        return 0;
}
\end{lstlisting}
}

\begin{frame}{main()}
  \lstI
\end{frame}

\defverbatim[colored]\lstI{
\begin{lstlisting}[language=C,basicstyle=\ttfamily]
static void sef_local_startup(void) {
        /* Register init callbacks. */
        sef_setcb_init_fresh(sef_cb_init);
        sef_setcb_init_lu(sef_cb_init);
        sef_setcb_init_restart(sef_cb_init);

        /* Register live update callbacks. */
        sef_setcb_lu_state_save(
                           sef_cb_lu_state_save);

        /* Let SEF perform startup. */
        sef_startup();
}
\end{lstlisting}
}

\begin{frame}{sef\_local\_startup(void)}
  \lstI
\end{frame}


\defverbatim[colored]\lstI{
\begin{lstlisting}[language=C,basicstyle=\ttfamily]
static int
sef_cb_init(int type, sef_init_info_t *info) {

  if (type == SEF_INIT_LU) lu_state_restore();

  /* ... i2cdriver_reserve_address() ... */

  if (bmp085_init() != OK) return EXIT_FAILURE;

  if (type != SEF_INIT_LU) {
    if (i2cdriver_subscribe_bus_updates(bus) != OK)
      return EXIT_FAILURE;
    i2cdriver_announce(bus);
  }
  return OK;
}
\end{lstlisting}
}

\begin{frame}{sef\_cb\_init(int type, sef\_init\_info\_t *info)}
  \lstI
\end{frame}


\defverbatim[colored]\lstI{
\begin{lstlisting}[language=C,basicstyle=\ttfamily]
static int
sef_cb_lu_state_save(int result, int flags)
{
        ds_publish_u32("bus", bus, DSF_OVERWRITE);
        ds_publish_u32("address", address,
                                   DSF_OVERWRITE);
        return OK;
}
\end{lstlisting}
}

\begin{frame}{sef\_cb\_lu\_state\_save(int result, int flags)}
  \lstI
\end{frame}


\defverbatim[colored]\lstI{
\begin{lstlisting}[language=C,basicstyle=\ttfamily]
static int
lu_state_restore(void)
{
        /* Restore the state. */
        u32_t value;

        ds_retrieve_u32("bus", &value);
        ds_delete_u32("bus");
        bus = (int) value;

        ds_retrieve_u32("address", &value);
        ds_delete_u32("address");
        address = (int) value;

        return OK;
}
\end{lstlisting}
}

\begin{frame}{lu\_state\_restore(void)}
  \lstI
\end{frame}


\defverbatim[colored]\lstI{
\begin{lstlisting}[language=C,basicstyle=\ttfamily]
static int
bmp085_init(void)
{
        if (version_check() != OK)
                return EXIT_FAILURE;

        if (read_cal_coef() != OK)
                return EXIT_FAILURE;

        return OK;
}
\end{lstlisting}
}

\begin{frame}{bmp085\_init(void)}
  \lstI
\end{frame}

\defverbatim[colored]\lstI{
\begin{lstlisting}[language=C,basicstyle=\ttfamily]
static ssize_t
bmp085_read(devminor_t minor, u64_t position,
    endpoint_t endpt, cp_grant_id_t grant,
    size_t size, int flags, cdev_id_t id);

static void
bmp085_other(message * m, int ipc_status);

static struct chardriver bmp085_tab = {
        .cdr_read       = bmp085_read,
        .cdr_other      = bmp085_other
};
\end{lstlisting}
}

\begin{frame}{libchardriver callbacks}
  \lstI
\end{frame}



\defverbatim[colored]\lstI{
\begin{lstlisting}[language=C,basicstyle=\ttfamily]

  if (measure(&temperature, &pressure) != OK)
      return EIO;

  /* ... fill buffer with measurements ... */

  dev_size = (u64_t)strlen(buffer);
  if (position >= dev_size) return 0;
  if (position + size > dev_size)
      size = (size_t)(dev_size - position);

  r = sys_safecopyto(endpt, grant, 0,
     (vir_bytes)(buffer + (size_t)position),size);

  return (r != OK) ? r : size;
\end{lstlisting}
}

\begin{frame}{bmp085\_read(...)}
  \lstI
\end{frame}


\defverbatim[colored]\lstI{
  \begin{lstlisting}[language=C,basicstyle=\ttfamily]
/* ... */

/* trigger temperature reading */
if (i2creg_write8(bus_endpoint, address, CTRL_REG,
                  CMD_TRIG_T) != OK)
    return -1;

micro_delay(UDELAY_T); /* wait for sampling. */

/* read the uncompensated temperature */
if (i2creg_read16(bus_endpoint, address,
                  SENSOR_VAL_MSB_REG, &ut) != OK)
    return -1;

/* ... */

return OK;
\end{lstlisting}
}

\begin{frame}{measure(int32\_t * temperature, int32\_t * pressure)}
  \lstI
\end{frame}

\begin{frame}{Tips for Debugging and Minimizing Bugs}
  \begin{itemize}
  \item Make many small changes
  \item Compile and test after each change
  \item Lots of logging via \texttt{minix/log.h}
  \item Hardware debugger when really stuck
  \item Avoid hardware simulators when possible
  \end{itemize}
\end{frame}

\begin{frame}{Pre-submission Checklist}
  \begin{itemize}
  \item Does the code build without any compiler warnings?
  \item Does building Minix for the other platform still work?
  \item Is the driver stable, and does it word as advertised?
  \item Does the driver have a negligible impact on performance?
  \item Are the changes broken up into a series of small commits?
  \item Are there well written commit messages?
  \item Is the coding style consistent with the NetBSD coding style?
  \item Have you complied with all applicable licenses?
  \item Have you given proper credit to collaborators?
  \end{itemize}
\end{frame}

\begin{frame}{Submitting your changes}
  \begin{itemize}
  \item Ask for feedback on \texttt{minix-dev} Google Group
  \item Submit a pull request on github
  \item Respond to feedback and requests for changes
  \end{itemize}
\end{frame}


\begin{frame}{Questions?}
  \centering
  Questions?
\end{frame}

\begin{frame}{Contact Info}
\begin{block}{Internet Relay Chat}
tcort on irc.freenode.net
\end{block}
\begin{block}{E-Mail}
tcort@minix3.org
\end{block}
\begin{block}{Twitter}
@tomcort
\end{block}
\begin{block}{Website}
http://www.tomcort.com/
\end{block}
\end{frame}

\end{document}
